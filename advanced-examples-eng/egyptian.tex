\documentclass[ngerman,a4paper,11pt]{article}
\usepackage[scaled]{helvet}
\usepackage{txfonts}
\usepackage{mathptmx} 
\usepackage[T1]{fontenc}
\usepackage{moodle}
\usepackage[margin=2cm]{geometry}
\usepackage{python}
\usepackage{graphicx}
\begin{document}
\begin{quiz}{Egyptian Calculations}
\begin{python}
def egyptian(c,d):  
  i=1  
  j=2   
  L=[]  
  K=[]  
  print(r"$$\begin{array}{rrr}")  
  a=c  
  b=d  
  rest1=a
  zeile=0
  while i<=a:  
    L.append([i,i*b,zeile])  
    rest1-=i
    i=10*i
    zeile+=1
  while j-1<=rest1:  
    L.append([j,j*b,zeile])  
    j=2*j  
    rest=rest1
    zeile+=1
  rest=a
  L.sort(key=lambda x: int(x[0])) 
  for i in range(0,len(L)):  
    if rest>=L[len(L)-1-i][0]:  
      z=L[len(L)-1-i][0]  
      L[len(L)-1-i][0]='/ &'+str(L[len(L)-1-i][0])  
      rest=rest-z  
    else:  
      L[len(L)-1-i][0]=' & '+str(L[len(L)-1-i][0])  
  L.sort(key=lambda x: int(x[2]))
  for i in range(0,len(L)):  
    print(L[i][0],' & ',L[i][1],r"\\")  
    if (L[i][0]).find('/')!=-1:  
      K.append(L[i][1])  
  
  print("\hline")  
  
  print(" &",a," & ",sum(K))  
  print(r"\end{array}$$ ")
  return('')  
auswahl1=[[659,15],[479,18]]
auswahl2=[
  [1219,23],[1007,19],[1127,23],[1073,37]]
c=0
for y in auswahl1:
  for x in auswahl2:
    c+=1
    print(r"\begin{essay}[points=6, response format=html, response field lines=20, template={<h2>Part a.):</h2><p>","<span style=\"font-size: medium;\">",r"(Write down your solution here)<br><br></span></p><h2>Part b.):</h2>","<p><span style=\"font-family: \'courier new\', courier, monospace; font-size: medium;\">/1&nbsp;&nbsp;&nbsp;&nbsp;",x[1],"<br>(Write down the intermediate steps here)<br>--------<br>&nbsp;",x[0]//x[1],"&nbsp;",x[0],r"</span></p><h2>Part c.):</h2><p>","<span style=\"font-size: medium;\">",r"(Write down your solution here)<br><br></span></p>}]{Egytian Calculations (",c,r")}",sep='')
    print(rf"""
    \textbf{{Ancient Egytian Calculations}}\\<ol type=a><li>
    Use the following example to explain how the ancient Egyptian method of multiplying two numbers (in this case: ${y[1]}\cdot{y[0]}$) works. 
    {egyptian(y[1],y[0])}
    </li>\\
    <li>    
    Using the ancient Egyptian method, solve the division task ${x[0]}:{x[1]}$.
    The first and last line of the solution is already given below.
    </li>\\
    <li>
    Using $180:27$ as an example, explain why not all division tasks can be solved in this way, even if you allow unit fractions in the form $\frac{{1}}{{2^n}}$.
    </li></ol>\\ 
    \item This task has to be corrected manually, sorry!
    \end{{essay}}""")
    if c%2==0 and c<8:
      print(r"\newpage")
    else:
      print(r" \bigskip \,\\ \medskip ")
\end{python}	
\end{quiz}
% --- The following python script is necessary to replace quotation marks, which could not be written directly --- %
% --- For importing into moodle, please make sure to use the -ready version of the xml-file                    --- %
\begin{python} 
with open("egyptian-moodle.xml", "rt") as fin:
    with open("egyptian-moodle-ready.xml", "wt") as fout:
        for line in fin:
            fout.write(line.replace('&rdquo;', '\"').replace('&rsquo;', '\''))
\end{python}
\end{document}
\end{document}
